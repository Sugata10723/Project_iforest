\documentclass{css}
%\documentclass[english]{css}

\usepackage[dvips]{graphicx}
\usepackage{latexsym}

\def\|{\verb|}

\newcommand{\cssyear}[0]{2023}
\newcommand{\cssname}[0]{CSS 2023}
\newcommand{\cssversion}[0]{2023/06/01}
\newcommand{\cssemail}[0]{css2023-office@iwsec.org}

\begin{document}

%% 本文が和文の場合,タイトル・著者名・著者所属・概要は,和文・英文共に必須.
%% If you prepare this manuscript in English, there is no need to put Japanese metadata (title, author names, affiliations, abstract, and keywords) in it.

\title{Isolation Forestを用いた\\IoT向け異常検知手法に関する考察}
\etitle{A Study on Anomaly Detection Method \\for IoT using Isolation Forest}

\affiliate{XX}{東京工業大学 情報理工学院 数理・計算科学系 松浦研究室\\
Department of Mathematical and Computing Sciences, School of Computing, Tokyo Institute of Technology}
\affiliate{YY}{株式会社YYセキュリティ研究所\\
Security Laboratories, YY Corporation}
\paffiliate{ZZ}{国立研究開発法人ZZ研究所\\
National Institute of ZZ}

%% メールアドレスは省略可能だが,代表者のメールアドレスは必須.
%% 姓名の間は半角スペースを入れること.

\author{菅田 大輔}{Daisuke Sugata}{XX}[sugata.d.aa@m.titech.ac.jp]
\author{安全 花子}{Hanako Anzen}{XX, YY, ZZ}

%% the following is author command for english option.
%% at least one e-mail address is required.

%\author{Taro Joho}{XX}[taro.joho@xx.ac.jp]
%\author{Hanako Anzen}{XX, YY, ZZ}

\begin{abstract}
あああああああああああああああああああああああああああああああああああああああああああああああああああああああああああああああああああああああああああああああああああああああああああああああああああああああああああ

\end{abstract}

%% キーワード (1--5単語) の記載は任意.

\begin{jkeyword}
Isolation Forest, IoT, IDS, 異常検知
\end{jkeyword}

\begin{eabstract}
aaaaaaaaaaaaaaaaaaaaaaaaaaaaaaaaaaaaaaaaaaaaaaaa.
aaaaaaaaaaaaaaaaaaaaaaaaaaaaaaaaaaaaaaaaaaa.
aaaaaaaaaaaaaaaaaaaa.
aaaaaaaaaa.

\end{eabstract}

%% the following keyword part is optional and can be omitted.

\begin{ekeyword}
Isolation Forest, IoT, IDS, Anomaly Detection
\end{ekeyword}

%% if you use english opsion, you should put your English abstract in the abstract environment.
%% eabstract is not displayed in english mode.

\maketitle

%1
\section{はじめに}
以下のことを書く.

\begin{itemize}
    \item 大目標
    \item 大目標を実現する必要性
    \item 問題提起:大目標を達成するために必要なことを述べ,そのためにどのような問題があるのかを述べる(小目標に分割する).
    \item 関連研究:問題を解決するための従来研究を紹介
    \item 本研究の目的
\end{itemize}


\section{研究方法}

\subsection{Isolation Forestの説明}
Isolation Forestの説明

\subsection{IDSの概要}

IDSの概要をかく.


\section{結果と考察}

\subsection{実験方法}

\subsubsection{実験環境}
他の人が再現できるように実験環境を書く.

\subsubsection{データセット}
使用したデータセットの概要と, その妥当性について述べる.

\subsubsection{評価指標}
使用した評価指標と, その妥当性について述べる.

\subsection{結果}
実験の結果,得られるデータから読み取れる客観的事実を書く.
この時, 論文の目的を達成するためにどのような主張をどのような結果(データ)に基づいて説明すべきかを考える.

\subsection{考察}

\subsubsection{本論文における目的に即した結論を導く}
\begin{itemize}
    \item 本結果を一般化したどのような結論を導き出せるかを,論文の目的に即して述べる.
    \item 実験結果の妥当性を説明する.
\end{itemize}

\subsubsection{結果から予測される問題を提起する.}
\begin{itemize}
    \item 結果が生じた理由について考察する.
    \item 本実験結果を認めると,どのような現象の予測や応用可能性があるかを述べる.
\end{itemize}

\section{おわりに}
おわりにを書く.

\begin{acknowledgment}
謝辞を書く.
\end{acknowledgment}

\begin{thebibliography}{10}

\end{thebibliography}

\end{document}
